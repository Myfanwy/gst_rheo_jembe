%%%%%%%%%%%%%%%%%%%%%%%%%%%%%%%%%%%%%%%%%
% Professional Table
% LaTeX Template
% Version 1.0 (11/10/12)
%
% This template has been downloaded from:
% http://www.LaTeXTemplates.com
%
% License:
% CC BY-NC-SA 3.0 (http://creativecommons.org/licenses/by-nc-sa/3.0/)
%
% Note: to use this table in another LaTeX document, you will need to copy
% the \usepackage{booktabs} line to the new document and paste it before 
% \begin{document}. The table itself can then be pasted anywhere in the new
% document.
%
%%%%%%%%%%%%%%%%%%%%%%%%%%%%%%%%%%%%%%%%%

\documentclass{article}

\usepackage{booktabs} % Allows the use of \toprule, \midrule and \bottomrule in tables for horizontal lines
\usepackage[landscape, margin = 0.4in]{geometry} % changes whole document to landscape orientation with 0.5-in margins
\usepackage{longtable} % allows table to go for more than one page
\usepackage{amsmath} % for equations
\usepackage{listings} % for R code
\usepackage{xcolor} % for syntax highlighting
\begin{document}
% Listings Style
\newcommand\mystyle{\lstset{
language = R,
basicstyle = \ttfamily,
commentstyle=\color{gray},
 otherkeywords={map2stan, data},
 keywordstyle=\color{blue}
 }}
 % custom environment for your listings
 \lstnewenvironment{mylist}[1][]
 {
 \mystyle
 \lstset{#1}
 }
 {}
 %%%%%%%%%%%%%%%%%%%%%%%%%%%
 Experiment 2: Full Model Set and R Specification
\begin{longtable}{p{.10\textwidth} |p{.40\textwidth} |p{.50\textwidth} }
\toprule % Top horizontal line
Model & Formula & Rstan Code \\ % Column names row
\midrule % In-table horizontal line
%
%%%%%%%%%%%%%%%%%%
m0b & % Begin First Row
%%%%%%%%%%%%%%%
$$ Y_i \sim Beta(p_i, \theta) $$
$$ logit(p_i) = \alpha $$
$$ \alpha \sim Normal(0,1) $$
$$ \theta \sim HalfCauchy(0, 1) $$
%begin code - m0, Row 1
 &
 {\begin{mylist}
# Intercept Model
m0b <- map2stan(
  alist(
    y ~ dbeta2(p,theta),
    logit(p) <- a,
    a ~ dnorm(0, 1),
    theta ~ dcauchy(0,1)
  ),
  data = dlist,
  constraints=list(theta="lower=0"),
  start=list(theta = 1), warmup=1000, iter=1e4, cores=2 )
 \end{mylist}}
 \\ % End Content row 1
 %%%%%%%%%%%%%%%%%%%%%%%%%%%
 m1b &  % Begin Row 2
 $$ Y_i \sim Beta(p_i, \theta) $$
 $$ logit(p_i) = \alpha_{\textsc{fish}[i]} $$
 $$ \alpha_{\textsc{fish}[i]} \sim Normal(0,1) $$
 $$ \theta \sim HalfCauchy(0, 1) $$
&
% begin code - m1b, Row 2
{\begin{mylist} 
# Fixed effects model: fish
m1b <- map2stan(
  alist(
    y ~ dbeta2(p,theta),
    logit(p) <- a_fish[fish_id],
    a_fish[fish_id] ~ dnorm(0,1),
    theta ~ dcauchy(0,1)
  ),
  data=dlist,
  constraints=list(theta="lower=0"),
  start=list(theta=1), warmup=1000 , iter=1e4, cores=2)  \end{mylist}}
\\ % End content Row 2
%%%%%%%%%%%%%%%%%%%%%%%%%%%%%
m2b & % Begin Row 3
$$ Y_i \sim Beta(p_i, \theta) $$
$$ logit(p_i) = \alpha_{\textsc{fish}[i]} $$
$$ \alpha_{\textsc{fish}[i]} \sim Normal(\alpha,\sigma_\textsc{fish}) $$
$$ \alpha \sim Normal(0,1) $$
$$ \sigma_\textsc{fish} \sim HalfCauchy(0,1) $$
$$ \theta \sim HalfCauchy(0, 1) $$
&
% begin code - m2, Row 3
{\begin{mylist}
# Varying intercepts: fish
m2b <- map2stan(
  alist(
    y ~ dbeta2(p,theta),
    logit(p) <- a_fish[fish_id],
    a_fish[fish_id] ~ dnorm(a, sigma_fish),
    a ~ dnorm(0, 1),
    sigma_fish ~ dcauchy(0,1),
    theta ~ dcauchy(0,1)
  ),
  data = dlist,
  constraints=list(theta="lower=0"),
  start=list(theta=1), warmup=1000, iter=1e4, cores=2 )
  \end{mylist}}
\\ % End content Row 3
%%%%%%%%%%%%%%%%%%%%%%%%%%%%
m3b & % Begin Row 4
$$ Y_i \sim Beta(p_i, \theta) $$
$$ logit(p_i) = \alpha + \alpha_{\textsc{fish}[i]} + \beta_{\textsc{Treatment}_[i]} $$
$$ \alpha_{\textsc{fish}[i]} \sim Normal(0,\sigma_{\textsc{fish}}) $$
$$ \alpha \sim Normal(0, 1) $$
$$ \sigma_{\textsc{fish}} \sim HalfCauchy(0,1) $$
$$ \beta_{\textsc{Treatment}}  \sim Normal(0,1) $$
$$ \theta \sim HalfCauchy(0, 1) $$
&
% begin code - m3, Row 4
{\begin{mylist}
# varying intercepts for fish by (fixed) treatment effect
m3b <- map2stan(
  alist(
    y ~ dbeta2( p, theta ),
    logit(p) <- a + a_fish[fish_id] + 
    b_treatment*treatment ,
    a_fish[fish_id] ~ dnorm(0, sigma_fish),
    a ~ dnorm(0, 1),
    b_treatment ~ dnorm(0, 1),
    theta ~ dcauchy(0,1),
    sigma_fish ~ dcauchy(0, 1)
  ),
  data = dlist,
  constraints=list(theta="lower=0"),
  start=list(theta=1), warmup=1000 , iter=1e4, cores=2) 
  \end{mylist}}
\\ % end content row 4
%%%%%%%%%%%%%%%%%%%%%%%%%%%%%%%%%%
m4b & % Begin Row 5
$$ Y_i \sim Beta(p_i, \theta) $$
$$ logit(p_i) = \alpha + \alpha_{\textsc{fish}[i]} + \beta_{\textsc{Treatment}[i]} $$
$$ \alpha_{\textsc{fish}[i]} \sim Normal(0,\sigma_{\textsc{fish}}) $$
$$ \beta_{\textsc{Treatment}[i]} \sim Normal(0,\sigma_{\textsc{Treatment}}) $$
$$ \alpha \sim Normal(0, 1) $$
$$ \sigma_{\textsc{fish}} \sim HalfCauchy(0,1) $$
$$ \sigma_{\textsc{Treatment}} \sim HalfCauchy(0,1) $$
$$ \theta \sim Exp(1) $$
&
% begin code - m4, Row 5
{\begin{mylist}
# Varying Intercepts: fish and treatment
m4b <- map2stan(
  alist(
    #likelihood
    y ~ dbeta2( p, theta ),
    # linear model
    logit(p) <- a + a_fish[fish_id] + b_treat[treatment],
    # adaptive priors
    a_fish[fish_id] ~ dnorm(0,sigma_fish),
    b_treat[treatment] ~ dnorm(0, sigma_treat),
    # fixed priors
    a ~ dnorm(0, 1),
    theta ~ dexp(1),
    sigma_fish ~ dcauchy(0,1),
    sigma_treat ~ dcauchy(0,1)
  ),
  data = dlist,
  constraints=list(theta="lower=0"),
  start=list(theta=1), warmup=1000 , iter=1e4 , cores=2 )
  \end{mylist}} 
\\ % end content row 5
%%%%%%%%%%%%%%%%%%%%%%%%%%%%%%%%%%%%%
mb1NC & % Begin Row 6
$$ Y_i \sim Beta(p_i, \theta) $$
\small  
$$ logit(p_i) =  \beta_1{\textsc{fish}_[i]} Light_i  + \beta_2{\textsc{fish}_[i]} Dark_i $$
$$
\begin{bmatrix}
\beta_1{\textsc{fish}}\\
\beta_2{\textsc{fish}} \\
\end{bmatrix} \sim MVNormal \Big(
\begin{bmatrix}
\alpha \\
\beta_1 \\
\beta_2 \\
\end{bmatrix}, S \Big)
$$
$$ \beta_1  \sim Normal(0,1) $$
$$ \beta_2  \sim Normal(0,1) $$
$$ \sigma_{\textsc{fish}} \sim Exp(1) $$
$$ \theta \sim HalfCauchy(0, 1) $$
$$ \rho_{FISH} \sim LKJcorr(2) $$
&
% begin code - mb1NC, Row 6
{\begin{mylist}
m1bNC <- map2stan(
  alist(
    #likelihood
    y ~ dbeta2( p, theta ),
    
    # linear model
    logit(p) <- (b_light + bl_fish[fish_id])*light + 
    (b_dark + bd_fish[fish_id])*dark,
    
    # adaptive non-centered priors 
    c(bl_fish, bd_fish)[fish_id] ~ dmvnormNC(sigma_fish, 
    Rho_fish),
    
    # fixed priors
    c(b_light, b_dark) ~ dnorm(0,1),
    theta ~ dcauchy(0,1),
    sigma_fish ~ dexp(1),
    Rho_fish ~ dlkjcorr(2)
  ),
  # data
  data = dlist,
  constraints=list(theta="lower=0"),
  start=list(theta=1), warmup=1000, iter=1e4, cores=2)  \end{mylist}}
\\ % end content row 6
%%%%%%%%%%%%%%%%%%%%%%%%%%%%%%%%%%%%
mB1 (begin barbelectomy compare) & % content for Row 7
$$ Y_i \sim Beta(p_i, \theta) $$
\small $$ logit(p_i) =  \alpha_{\textsc{fish}[i]} + \beta_{\textsc{Treatment}[i]}$$
$$
\begin{bmatrix}
\alpha_{\textsc{fish}} \\
\beta_{\textsc{fish}}
\end{bmatrix} \sim MVNormal \Big(
\begin{bmatrix}
\alpha \\
\beta
\end{bmatrix}, S \Big)
$$
$$
\mathbf{S} = \left( \begin{array}{cc}
\sigma_\alpha & 0 \\
0 & \sigma_\beta
\end{array} \right) R \left( \begin{array}{cc}
\sigma_\alpha & 0 \\
0 & \sigma_\beta
\end{array} \right)
$$
$$ \alpha  \sim Normal(0,1) $$
$$ \beta  \sim Normal(0,1) $$
$$ \sigma_{\textsc{fish}} \sim Exp(1) $$
$$ \theta \sim HalfCauchy(0, 1) $$
$$ \rho_{\textsc{fish}} \sim LKJCorr(2) $$
&
{\begin{mylist} % Begin code for mB1, Row 7
# varying intercepts with fish, fixed treatment effect
mB1 <- map2stan(
    alist(
    #likelihood
    y ~ dbeta2( p, theta ),
    
    # linear model
    logit(p) <- a_fish[fish_id] + b_treatment*treatment ,
    
    # adaptive NON-CENTERED priors 
    a_fish[fish_id] ~ dmvnormNC(sigma_fish, Rho_fish),
    
    # fixed priors
    b_treatment ~ dnorm(0,1),
    theta ~ dcauchy(0,1),
    sigma_fish ~ dexp(1),
    Rho_fish ~ dlkjcorr(2)
  ),
  # data
  data = dlistB,
  constraints=list(theta="lower=0"),
  start=list(theta=1), warmup=1000, iter=1e4, cores=2) \end{mylist}}
\\ % end content row 7
%%%%%%%%%%%%%%%%%%%%%%%%%%%%%%%%%%%%
mB2 & % content for Row 8
$$ Y_i \sim Beta(p_i, \theta) $$
\small $$ logit(p_i) =  \alpha_{\textsc{fish}[i]} + \beta_{\textsc{fish}[i]}Treatment$$
$$
\begin{bmatrix}
\alpha_{\textsc{fish}} \\
\beta_{\textsc{fish}}
\end{bmatrix} \sim MVNormal \Big(
\begin{bmatrix}
\alpha \\
\beta
\end{bmatrix}, S \Big)
$$
$$
\mathbf{S} = \left( \begin{array}{cc}
\sigma_\alpha & 0 \\
0 & \sigma_\beta
\end{array} \right) R \left( \begin{array}{cc}
\sigma_\alpha & 0 \\
0 & \sigma_\beta
\end{array} \right)
$$
$$ \alpha  \sim Normal(0,1) $$
$$ \beta  \sim Normal(0,1) $$
$$ \sigma_{\textsc{fish}} \sim Exp(1) $$
$$ \theta \sim HalfCauchy(0, 1) $$
$$ \rho_{\textsc{fish}} \sim LKJCorr(2) $$
&
{\begin{mylist} % Begin code for mB2, Row 8
#  allowing treatment slope to vary by fish
mB2 <- map2stan(
  alist(
    #likelihood
    y ~ dbeta2( p, theta ),
    
    # linear model
    logit(p) <- a_fish[fish_id] + (b_treatment + 
    b_fish[fish_id])*treatment ,
    
    # adaptive NON-CENTERED priors 
    c(a_fish, b_fish)[fish_id] ~ dmvnormNC(sigma_fish, 
    Rho_fish),
    
    # fixed priors
    b_treatment ~ dnorm(0,1),
    theta ~ dcauchy(0,1),
    sigma_fish ~ dexp(1),
    Rho_fish ~ dlkjcorr(2)
  ),
  # data
  data = dlistB,
  constraints=list(theta="lower=0"),
  start=list(theta=1), warmup=1000, iter=1e4, cores=2) \end{mylist}}
\\ % end content row 8
%%%%%%%%%%%%%%%%%%%%%%%%%%%%%%%%%%%%
mB3 & % content for Row 9
$$ Y_i \sim Beta(p_i, \theta) $$
\small $$ logit(p_i) =  \alpha_{\textsc{fish}[i]} + \beta_1{Light} + \beta_2{Dark}$$
$$\alpha_{\textsc{fish}[i]} \sim Normal(0, \sigma_{\textsc{fish}}) $$
$$ \alpha  \sim Normal(0,1) $$
$$ (\beta_1, \beta_2)  \sim Normal(0,1) $$
$$ \sigma_{\textsc{fish}} \sim Exp(1) $$
$$ \theta \sim HalfCauchy(0, 1) $$
&
{\begin{mylist} % Begin code for mB3, Row 9
#  varying intercepts on fish, individual treatment fixed effects
mB3 <- map2stan(
  alist(
    #likelihood
    y ~ dbeta2( p, theta ),
    
    # linear model
    logit(p) <- a_fish[fish_id] + b_light*light + 
    b_dark*dark ,
    
    # adaptive NON-CENTERED priors 
    a_fish[fish_id] ~ dnorm(0, sigma_fish),
    
    # fixed priors
    c(b_light, b_dark) ~ dnorm(0,1),
    theta ~ dcauchy(0,1),
    sigma_fish ~ dexp(1)
  ),
  # data
  data = dlistB,
  constraints=list(theta="lower=0"),
  start=list(theta=1), warmup=1000, iter=1e4, cores=2) \end{mylist}}
\\ % end content row 9
%%%%%%%%%%%%%%%%%%%%%%%%%%%%
mB4& % content for Row 10
$$ Y_i \sim Beta(p_i, \theta) $$
\small $$ logit(p_i) =  \alpha_{\textsc{fish}[i]} + \beta_{\textsc{2fish[i]}}Light_i + \beta_{\textsc{2fish[i]}}Dark_i$$
$$
\begin{bmatrix}
\alpha_{\textsc{fish}} \\
\beta_1{\textsc{fish}} \\
\beta_2{\textsc{fish}} \\
\end{bmatrix} \sim MVNormal \Big(
\begin{bmatrix}
\alpha \\
\beta_1 \\
\beta_2 \\
\end{bmatrix}, S \Big)
$$
$$ \alpha  \sim Normal(0,1) $$
$$ (\beta_1, \beta_2)  \sim Normal(0,1) $$
$$ \sigma_{\textsc{fish}} \sim Exp(1) $$
$$ \theta \sim HalfCauchy(0, 1) $$
$$ \rho_{\textsc{fish}} \sim LKJCorr(2) $$
&
{\begin{mylist} % Begin code for mB4, Row 10

# varying intercepts and slopes by fish and treatment
mB4 <- map2stan(
  alist(
    #likelihood
    y ~ dbeta2( p, theta ),
    
    # linear model
    logit(p) <- a_fish[fish_id] + 
    (b_light+ bl_fish[fish_id])*light + 
    (b_dark + bd_fish[fish_id])*dark ,
    
    # adaptive NON-CENTERED priors 
    c(a_fish, bl_fish, bd_fish)[fish_id] ~ 
    dmvnormNC(sigma_fish, Rho_fish),
    
    # fixed priors
    c(b_light, b_dark) ~ dnorm(0,1),
    theta ~ dcauchy(0,1),
    sigma_fish ~ dexp(1),
    Rho_fish ~ dlkjcorr(2)
  ),
  # data
  data = dlistB,
  constraints=list(theta="lower=0"),
  start=list(theta=1), warmup=1000, iter=1e4, cores=2) \end{mylist}}
\\ % end content row 10
%%%%%%%%%%%%%%%%%%%%%%%%%%%%
mB5& % content for Row 11
$$ Y_i \sim Beta(p_i, \theta) $$
\small $$ logit(p_i) =  \alpha_{\textsc{fish}[i]} + \beta_{\textsc{Barbels}[i]}$$
$$
\begin{bmatrix}
\alpha_{\textsc{fish}} \\
\end{bmatrix} \sim MVNormal \Big(
\begin{bmatrix}
\alpha \\
\end{bmatrix}, S \Big)
$$
$$ (\beta_{\textsc{Barbels}[i]})  \sim Normal(0,1) $$
$$ \sigma_{\textsc{fish}} \sim Exp(1) $$
$$ \theta \sim HalfCauchy(0, 1) $$
$$ \rho_{\textsc{fish}} \sim LKJCorr(2) $$
&
{\begin{mylist} % Begin code for mB5, Row 11

# Adding barbels as a fixed effect
mB5 <- map2stan(
  alist(
    #likelihood
    y ~ dbeta2( p, theta ),
    
    # linear model
    logit(p) <- a_fish[fish_id] + b_barbels*barbels ,
    
    # adaptive NON-CENTERED priors 
    a_fish[fish_id] ~ dmvnormNC(sigma_fish, Rho_fish),
    
    # fixed priors
    b_barbels ~ dnorm(0,1),
    theta ~ dcauchy(0,1),
    sigma_fish ~ dexp(1),
    Rho_fish ~ dlkjcorr(2)
  ),
  # data
  data = dlistB,
  constraints=list(theta="lower=0"),
  start=list(theta=1), warmup=1000, iter=1e4, cores=2) \end{mylist}}
\\ % end content row 11
%%%%%%%%%%%%%%%%%%%%%%%%%%%%
mB6& % content for Row 12
$$ Y_i \sim Beta(p_i, \theta) $$
\small $$ logit(p_i) =  \alpha_{\textsc{fish}[i]} + \beta_{\textsc{Barbels}[i]} + \beta_{\textsc{Treatment}[i]}$$
$$
\begin{bmatrix}
\alpha_{\textsc{fish}} \\
\end{bmatrix} \sim MVNormal \Big(
\begin{bmatrix}
\alpha \\
\end{bmatrix}, S \Big)
$$
$$ (\beta_{\textsc{Barbels}[i]}, \beta_{\textsc{Treatment}[i]})  \sim Normal(0,1) $$
$$ \sigma_{\textsc{fish}} \sim Exp(1) $$
$$ \theta \sim HalfCauchy(0, 1) $$
$$ \rho_{\textsc{fish}} \sim LKJCorr(2) $$
&
{\begin{mylist} % Begin code for mB6, Row 12

#  Adding barbels and treatment as fixed effects
mB6 <- map2stan(
  alist(
    #likelihood
    y ~ dbeta2( p, theta ),
    
    # linear model
    logit(p) <- a_fish[fish_id] + b_barbels*barbels + 
    b_treatment*treatment,
    
    # adaptive NON-CENTERED priors 
    a_fish[fish_id] ~ dmvnormNC(sigma_fish, Rho_fish),
    
    # fixed priors
    c(b_barbels, b_treatment) ~ dnorm(0,1),
    theta ~ dcauchy(0,1),
    sigma_fish ~ dexp(1),
    Rho_fish ~ dlkjcorr(2)
  ),
  # data
  data = dlistB,
  constraints=list(theta="lower=0"),
  start=list(theta=1), warmup=1000, iter=1e4, cores=2) \end{mylist}}
\\ % end content row 12
%%%%%%%%%%%%%%%%%%%%%%%%%%%%
mB7& % content for Row 13
$$ Y_i \sim Beta(p_i, \theta) $$
\small $$ logit(p_i) =  \alpha_{\textsc{fish}[i]} + \beta_{\textsc{1Treatment}[i]}Barbels_i$$
$$
\begin{bmatrix}
\alpha_{\textsc{fish}} \\
\beta_{\textsc{Treatment}} \\
\end{bmatrix} \sim MVNormal \Big(
\begin{bmatrix}
\alpha \\
\beta \\
\end{bmatrix}, S \Big)
$$
$$
\mathbf{S} = \left( \begin{array}{cc}
\sigma_\alpha & 0 \\
0 & \sigma_\beta
\end{array} \right) R \left( \begin{array}{cc}
\sigma_\alpha & 0 \\
0 & \sigma_\beta
\end{array} \right)
$$
$$ \beta_1  \sim Normal(0,1) $$
$$ (\sigma_{\textsc{treatment}}, \sigma_{\textsc{fish}}) \sim Exp(1) $$
$$ \theta \sim HalfCauchy(0, 1) $$
$$ \rho_{\textsc{fish}} \sim LKJCorr(2) $$
&
{\begin{mylist} % Begin code for mB7, Row 13

#  Allowing barbel effect slope to vary with treatment
mB7 <- map2stan(
  alist(
    #likelihood
    y ~ dbeta2( p, theta ),
    
    # linear model
    logit(p) <- a_fish[fish_id] + (b_barbels + 
    b_treatment[treatment])*barbels ,
    
    # adaptive NON-CENTERED priors 
    a_fish[fish_id] ~ dmvnormNC(sigma_fish, Rho_fish),
    b_treatment[treatment] ~ 
    dmvnormNC(sigma_treatment, Rho_treatment),
    
    # fixed priors
    b_barbels ~ dnorm(0,1),
    theta ~ dcauchy(0,1),
    c(sigma_fish, sigma_treatment) ~ dexp(1),
    c(Rho_fish, Rho_treatment) ~ dlkjcorr(2)
  ),
  # data
  data = dlistB,
  constraints=list(theta="lower=0"),
  start=list(theta=1), warmup=1000, iter=1e4, cores=2) \end{mylist}}
  \\ % end content row 13
  %%%%%%%%%%%%%%%%%%%%%%%%%%%%
mB8& % content for Row 13
$$ Y_i \sim Beta(p_i, \theta) $$
\small $$ logit(p_i) =  \alpha_{\textsc{fish}[i]} + \beta_{\textsc{fish}[i]}Barbels_i$$
$$
\begin{bmatrix}
\alpha_{\textsc{fish}} \\
\beta_{\textsc{fish}} \\
\end{bmatrix} \sim MVNormal \Big(
\begin{bmatrix}
\alpha \\
\beta \\
\end{bmatrix}, S \Big)
$$
$$
\mathbf{S} = \left( \begin{array}{cc}
\sigma_\alpha & 0 \\
0 & \sigma_\beta
\end{array} \right) R \left( \begin{array}{cc}
\sigma_\alpha & 0 \\
0 & \sigma_\beta
\end{array} \right)
$$
$$ \beta \sim Normal(0,1) $$
$$ \sigma_{\textsc{fish}} \sim Exp(1) $$
$$ \theta \sim HalfCauchy(0, 1) $$
$$ \rho_{\textsc{fish}} \sim LKJCorr(2) $$
&
{\begin{mylist} % Begin code for mB8, Row 14

#  Allowing barbel effect slope to vary with fish
mB8 <- map2stan(
  alist(
    #likelihood
    y ~ dbeta2( p, theta ),
    
    # linear model
    logit(p) <- a_fish[fish_id] + (b_barbels + 
    b_fish[fish_id])*barbels ,
    
    # adaptive NON-CENTERED priors 
    c(a_fish, b_fish)[fish_id] ~ 
    dmvnormNC(sigma_fish, Rho_fish),
    
    # fixed priors
    b_barbels ~ dnorm(0,1),
    theta ~ dcauchy(0,1),
    sigma_fish ~ dexp(1),
    Rho_fish ~ dlkjcorr(2)
  ),
  # data
  data = dlistB,
  constraints=list(theta="lower=0"),
  start=list(theta=1), warmup=1000, iter=1e4, cores=2) \end{mylist}}
  \\ % end content row 14
%%%%%%%%%%%%%%%%%%%%%%%%%%%%%%%%%%%%%%%%%
%\midrule % In-table horizontal line
%\bottomrule % Bottom horizontal line
%\end{tabular}
%\caption* % Table caption, can be commented out if no caption is required
%\label{tab:template} % A label for referencing this table elsewhere, references are used in text as \ref{label}
\end{longtable}
%\end{table}
\end{document}